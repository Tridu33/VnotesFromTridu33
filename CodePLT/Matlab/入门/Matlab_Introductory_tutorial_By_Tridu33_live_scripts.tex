% This LaTeX was auto-generated from MATLAB code.
% To make changes, update the MATLAB code and export to LaTeX again.

\documentclass{article}

\usepackage[utf8]{inputenc}
\usepackage[T1]{fontenc}
\usepackage{lmodern}
\usepackage{graphicx}
\usepackage{color}
\usepackage{listings}
\usepackage{hyperref}
\usepackage{amsmath}
\usepackage{amsfonts}
\usepackage{epstopdf}
\usepackage{matlab}

\sloppy
\epstopdfsetup{outdir=./}
\graphicspath{ {./Matlab_Introductory_tutorial_By_Tridu33_live_scripts_images/} }

\begin{document}

\subsection*{\href{http://www.cc.ntu.edu.tw/chinese/epaper/0037/20160620_3710.html}{MATLAB Live scripts}.mlx}

\begin{par}
\begin{flushleft}
\texttt{\underline{\textit{\textbf{上述}}}}是使用自带的\textit{webBroswe}r打开网页超链接
\end{flushleft}
\end{par}

\begin{par}
\begin{flushleft}
$Live scripts$,什么语言都能打。完成程式碼的撰寫與文字的編輯後,可以選擇將檔案完整輸出成PDF或是HTML的檔案類型,也可以另存成原本Editor使用的m-file格式,只不過文字格式和額外插入的圖片都會消失。类似的文学编程的操作很多,比如Apple的playground(2016),Mathematica的互动环境,Python 的Jupyternotebooks。
\end{flushleft}
\end{par}

\begin{par}
\begin{flushleft}
$\lambda$数学公式不在话下,$Latex$也不是不可以。
\end{flushleft}
\end{par}

\vspace{1em}


\begin{par}
\begin{center}
これわ3D Model Of Tridu33
\end{center}
\end{par}

\begin{par}
\begin{center}
\includegraphics[width=\maxwidth{18.063221274460613em}]{image_0}
\end{center}
\end{par}

\begin{par}
\begin{flushleft}
Matlab,一张万能的数学演算纸,冥冥中,世界运作的规律就躺在你的符号数据之间。
\end{flushleft}
\end{par}

\begin{par}
\begin{flushleft}
闲话休提,言归正传,书接上一回。Mtlab的入门和备忘最好的方法就是姑且当个Ctrl+cv程序员,把必要的一些代码都记下来以后知道怎么找,怎么用。
\end{flushleft}
\end{par}

\begin{par}
\begin{flushleft}
1、 matlab软件的界面以及基本的一些操作,比如如何在command window输入命令。还有一些最基本的函数:clc,clear,who,whos,“;”的用法以及输出格式控制等,和txt,Exceli数据交互,甚至和Excel的VBA混合编程的东西。
\end{flushleft}
\end{par}

\begin{matlabcode}
%代码段
% clear
close all
clc
%% matlab至此,匹夫复何求?
cd;
%注释pathtool命令调用窗口选择文件路径


\end{matlabcode}

\begin{par}
\begin{flushleft}
2、 有哪些数据类型和算法,字符,数值,cell和structure。数值和字符串之间的转换,数值计算语法。
\end{flushleft}
\end{par}

\begin{matlabcode}
%
\end{matlabcode}

\begin{par}
\begin{flushleft}
3、 数组,矩阵及常用的操作,如何创建一维矩阵,多维矩阵,以及线性代数中矩阵相加、相乘、分解等操作。matlab在数据类型方面的操作确实非常强大和灵活,比如创建一个不定长数组,创建一个等差数列。都非常方便,用好矩阵的描述手法,在Matlab中用数学演算世界运作的规律。
\end{flushleft}
\end{par}


\begin{matlabcode}
%点选上述wsectionBreak就可以区分区块
%这部分是基本操作,数据,矩阵和一些常见运算
disp("Hello-world");
\end{matlabcode}
\begin{matlaboutput}
Hello-world
\end{matlaboutput}
\begin{matlabcode}
%Create a matrix
M1=[1,1;2,2]
\end{matlabcode}
\begin{matlaboutput}
M1 = 
     1     1
     2     2

\end{matlaboutput}
\begin{matlabcode}
%create a complex matrix
M2=[1+1i 1+1i;2+2i 2+2i];
\end{matlabcode}

\begin{par}
\begin{flushleft}
4、 求解简单的方程,线性方程组,微分和积分,从牛顿的角度窥探世界运作的规律。
\end{flushleft}
\end{par}

\begin{par}
\begin{flushleft}
脚本文件和函数文件------脚本是不需要输入参数的,函数文件需要输入参数还有返回值.。符号计算。逻辑计算。内嵌函数等技巧蛋。@表示匿名函数,或者叫做“lambda表达式”。Lambda演算等
\end{flushleft}
\end{par}

\vspace{1em}

\begin{matlabcode}
M1*M2
\end{matlabcode}
\begin{matlaboutput}
ans = 
   3.0000 + 3.0000i   3.0000 + 3.0000i
   6.0000 + 6.0000i   6.0000 + 6.0000i

\end{matlaboutput}

\begin{par}
\begin{flushleft}
5、 最基本的绘图函数,如何绘制线形图,条形图,如何为横坐标和纵坐标加上相应的标题,如何设置坐标系,如何设置统计图的标题等。以及3D绘图,不过我现在还没有需要用3D绘图的的地方。
\end{flushleft}
\end{par}

\begin{par}
\begin{flushleft}
因为Matlab绘图功能出了名地丑陋。进阶的画图工具包使用等等。进阶计算机图形学,做动画,做视频等。
\end{flushleft}
\end{par}

\begin{matlabcode}
x=1:0.1:10;
y=sin(x);
plot(x,y,['r-']);
\end{matlabcode}
\begin{center}
\includegraphics[width=\maxwidth{56.196688409433015em}]{figure_0}
\end{center}

\begin{par}
\begin{flushleft}
6、 If-else,for,while,swith-case这些逻辑控制语句的含义和基本用法,跟着官方文档和博客中的案例敲代码。Mtalab面向对象程序设计,进阶GUI。函数句柄,面向对象。视频音频处理。图像处理,计算机视觉相关算法。
\end{flushleft}
\end{par}

\begin{matlabcode}
%
\end{matlabcode}

\begin{par}
\begin{flushleft}
Matlab是编程语言,图灵完备,对于可解问题无所不能,也可以其他语言混合编程,C,Python。。。
\end{flushleft}
\end{par}

\begin{par}
\begin{flushleft}
Matlab进阶其实就是安装或修改Matlab的可以选的哪些模块,简单来说,跟自己写的子函数脚本算法一个意思,大牛写好了我们直接调用。但是数据结构也要学才能灵活调用,所以我们也要学习那些相关的算法是怎么来的,接口怎么用。(算法就是大牛设计好的写好的RAM机的套路操作,面向对象,留个接口句柄参数。。。)
\end{flushleft}
\end{par}

\vspace{1em}

\begin{matlabcode}
%
\end{matlabcode}

\begin{par}
\begin{flushleft}
7 . 爬虫,数据分析,统计学习相关工具包。
\end{flushleft}
\end{par}

\begin{matlabcode}
%
\end{matlabcode}

\begin{par}
\begin{flushleft}
9.数学实验,物理可视化。数值积分,计算方法相关仿真建模。进阶工程应用有Simulink,S-Function,机械控制,信号处理,航空航天相关工具包,
\end{flushleft}
\end{par}

\begin{matlabcode}
%
\end{matlabcode}

\begin{par}
\begin{flushleft}
10. 算法研究,计算机视觉,神经网络,深度学习,机器学习等智能算法。 
\end{flushleft}
\end{par}

\begin{matlabcode}
%
\end{matlabcode}

\begin{par}
\begin{flushleft}
8.Pretty等幺蛾子有趣彩蛋,很短的ctrl+CV代码,有趣,有用,也没用,不得不单独拿出来说的就是,当命令行相关工具包成熟了,集大成者就会开发一个GUI界面让更多人0基础上手。
\end{flushleft}
\end{par}

\begin{matlabcode}
%
\end{matlabcode}

\begin{par}
\begin{flushleft}
附录
\end{flushleft}
\end{par}

\begin{par}
\begin{flushleft}
Matlab命令大全
\end{flushleft}
\end{par}

\begin{par}
\begin{flushleft}
进阶技巧,算法作图工具包,相关搜索资源。
\end{flushleft}
\end{par}

\begin{par}
\begin{flushleft}
行业相关的工具箱。
\end{flushleft}
\end{par}

\begin{par}
\begin{flushleft}
进阶彩蛋幺蛾子。
\end{flushleft}
\end{par}


\vspace{1em}

\begin{par}
\begin{flushleft}
杠精上脑, 道理大家都明白,可是上述这些代码,官网Doc,help+命令,像这样的[学习网站](https://www.w3cschool.cn/matlab/matlab-5use28gb.html)要多少有多少?
\end{flushleft}
\end{par}

\begin{par}
\begin{flushleft}
这个仓库重复记载这些规律吗?那还不如看官方Doc网站翻译。只有菜鸟才喜欢重复造轮子,抄来抄去,搬砖。一些牛人都喜欢做出来一些,这个世界以前没有人想做,没有人做过的东西。
\end{flushleft}
\end{par}

\section*{**我希望把这里协同共建一个高效检索,分类明确的Matlab解决问题高校检索的数据库目录**}

\begin{par}
\begin{flushleft}
这也放的是每个主题下最基本能活下去的基本操作,对应的每个\%**.mlx文件才是深入耕耘的代码笔记。
\end{flushleft}
\end{par}

\vspace{1em}

\begin{par}
\begin{flushleft}
这里可以是我们(如果有人闲着没事PR),如果我遇到某个问题,我需要的不是一本冷冰冰的书,用的时候从头翻到脚找到自己想要的,我需要的是能搜索,精简,想找相关实现功能,拿来就能用的“知识轮子”,一个记笔记和自己以后找资料的地方!一个文学变成,高效学习的地方。
\end{flushleft}
\end{par}

\begin{par}
\begin{flushleft}
资源放错地方就是垃圾,互联网是一个分布式开源数据库,好的教程不应该是流量闭环,应该指向一条一辈子都做不完的康庄大道。。。
\end{flushleft}
\end{par}

\begin{par}
\begin{flushleft}
万一假如如果你有精彩内容想合作的可以PR链接到这儿给大家看的可以链接到相关的内容。自媒体推广,广告之类的我会拒绝PR。
\end{flushleft}
\end{par}


\end{document}
